\textsf{\textbf{\Large{}\hspace{20pt}}}\textsf{\textbf{\LARGE{}Terminologický
slovník}}\vspace{13pt}

\bigskip

\begin{itemize}
	\item Markup
\item Vaporware
\item Počítačový robot
\item Hypertext
\item X.500
\item FTP Server
\item Fulltextové vyhledávání
\item USENET
\item WHOIS
\item Klíčové slovo
\item Meta informace
\item Hlavička webu
\item Webmaster
\item Outsourcing
\item Rekurzivní algoritmus
\item Fuzzy logika
\item Ranking
\item Šedá skříňka
\item Backlink
\item HTTPS
\item Rastrový obrázek
\item robots.txt 
\item eCommerce
\item Obrázkový slider
\item Shortcode
\item OFF-PAGE - Soubor technik/prací za účelem získávání zpětných odkazů - rozšíření odkazového portfolia, lepsí pozice na daná klíčová slova ve vyhledávačích
\item ON-PAGE - Soubor technik/prací za účelem zkvalitnění obsahu stránek, zlepšení jejich struktury a přístupnosti jak pro vyhledávače (zlepšení pozic ve vyhledávačích), tak pro návštěvníky
\item LINKBUILDING - Práce za účelem získání zpětného odkazu - jedná se například o psací do diskusí/fór a sociálních sítí za účelem získání zpětných odkazů (primární účel: zpětný odkaz, sekundární: noví návštěvníci)
\item SEO – Search Engine Optimization – Optimalizace pro vyhledávače.
\item SEM – Search Engine Marketing – Marketing ve vyhledávačích (Sklik, Adwords).
\item PPC – Pay per Click – Platba za proklik. PPA – Pay per Action – Platba za akci (konverzi).
\item CPM – Cost per Mile – Cena za 1000 zobrazení.
\item CTR – Click–through rate – Míra prokliku, vyjadřuje poměr kliknutí k počtu zobrazení.
\item SERP – Search engine results page – Stránka s výsledky vyhledávání.
\item Adwords – PPC reklamní systém od společnosti Google.
\item Sklik – PPC reklamní systém od společnosti Seznam.
\item URL – Uniform Resource Locator – Internetová adresa webové stránky.
\item Tagy (elementy) – Značky, které se používají při psaní (x)html kódu.
\item Sémantika – Stylistická pravidla pro správné využití (x)html elementů.
\item Zpětné odkazy – Back links – Zpětný odkaz je odkaz, který vede na web z jiného webu. Množství a kvalita zpětných odkazů ovlivňuje off-page faktory webu. Odkazy mají největší vliv na hodnotu ranku vyhledávačů.
\item ROI – Return on Investment – Návratnost investice. Jedná se o poměr zisku a investované částky.
\item Metadata – Informace uložené v hlavičce webu.
\item Copywriting – Psaní textu pro web.
\item Keyword – Klíčové slovo.
\item PR – Page Rank – Jedná se o interní hodnotu vyhledávače Google, podle které Google určuje kvalitu webu. Vyhledávače předpokládají, že na kvalitnější weby se bude více odkazovat a tak webu, na který směřuje hodně kvalitních zpětných odkazů, přiřadí vysokou hodnotu PR. PR je jedna z mnoha veličin, podle které Google řadí výsledky vyhledávání. Na rozdíl od GTPR je aktualizován průběžně.
\item GTPR – Google Toolbar Page Rank – Jedná se o výstupní hodnotu vyhledávače Google. GTPR je aktualizován jednou za 3 – 6 měsíců. Protože skutečný PR nejde zjistit, tak se GTPR používá jako jedna z veličin, podle které se odvíjí cena zpětných odkazů.
\item Srank – Rank, který používá vyhledávač Seznam.
\item Anchor text – Viditelný text u odkazu. Cílový web tím posiluje pozici ve vyhledávači na klíčové slovo obsažené v anchor textu.
\item Mapa webu – Jedná se o hierarchicky seřazený seznam všech stránek webu. Pomáhá vyhledávači najít stránky, které jsou umístěné hluboko ve stromové struktuře webu.
\item TLD – Top Level Domain – Domena první úrovně (např .cz).
\item CSS – Cascading Style Sheets – Kaskádové styly.
\item Black Hat SEO – Nepovolené praktiky SEO, které vyhledávače postihují.
\item Google Toolbar – Toolbar v prohlížeči od společnosti Google, který je využíván k hledání ve vyhledávači Google, nebo pro zjištění GTPR webu.
\item Keyword stuffing – Nesmyslné nadměrné používání klíčových slov v textu webu. Patří mezi nepovolené praktiky SEO a vyhledávače mohou takovéto weby penalizovat.
\item Long Tail – Dlouhý ocas – Často vyhledávaná slova tvoří jen malou část z celkových dotazů ve vyhledávači. Většina vyhledávaných frází je hledána třeba jen jednou nebo málokdy. Většinu dotazů ve vyhledávači tvoří právě málo vyhledávané fráze.
\item Navigační dotaz – Navigational query – Hledání webu pomocí vyhledávání doménového jména ve vyhledávači.
\item UIP – Počet návštěvníků s unikátní IP adresou za určité období.
\item Visit – Návštěva – Většina počítadel započítává jednu návštěvu jednoho návštěvníka max. jednou za 30 minut. Impression – Imprese – Udává počet zobrazení (např. počet zobrazení reklamního banneru).
\item Click fraud – Neplatné kliknutí – Podvodné klikání partnera u PPC obsahové sítě.
\item Validita – Aby byl web validní, musí dodržovat předepsané standarty (např. W3C).
\item W3C – World Wide Web Consortium – Mezinárodní konsorcium, které vyvíjí webové standarty.
\item Fulltext – Vyhledávání v indexu stránek.
\item Katalog – Seznam webů uspořádaných do kategorií.
\item Description – Meta tag description by měl obsahovat stručný popis dané stránky webu.
\item ODP – Open Directory Project – http://www.dmoz.org katalog webových stránek
\item Konverze – Provedení určité akce na webu – např. nakoupení zboží, přihlášení se k odběru novinek apod.
\item Subdoména – Doména 3., nebo nižšího řádu.
\item Traffic – Návštěvnost.
\item Volná shoda – Veškeré dotazy ve vyhledávačích, které obsahují dané klíčové slovo.
\item Přesná shoda – Patří sem veškeré dotazy ve vyhledávačích, které jsou hledány pomocí přesného znění hledané fráze.
\end{itemize}
