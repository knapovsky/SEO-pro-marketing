\AtBeginDocument{
  \def\labelitemiii{\Pisymbol{psy}{183}}
}

\makeatother

\begin{document}
\thispagestyle{empty}
%export LANG=cs_CZ

\begin{center}
\textsf{\textbf{\large{}VYSOKÁ ŠKOLA EKONOMICKÁ V PRAZE}}\textsf{\textbf{}}\\
\textsf{\textbf{\large{}Fakulta informatiky a statistiky}}\\
\textsf{\large{}Katedra informačních technologií}
\par\end{center}{\large \par}

\begin{center}
\textsf{Studijní program: Aplikovaná informatika}\textsf{\textbf{\large{}}}\\
\textsf{Obor: Kognitivní informatika}
\par\end{center}

\vfill

\begin{center}
\includegraphics[scale=0.8]{img/logo_vse}
\par\end{center}

\vfill

\begin{center}
\textsf{\textbf{\Large{}DIPLOMOVÁ PRÁCE}}
\par\end{center}{\Large \par}

\textsf{\medskip}

\begin{center}
\textsf{\textbf{\large{}Aplikace SEO technik pro marketing}}\textsf{\large{}}\\
\textsf{Application of SEO techniques for marketing}
\par\end{center}

\begin{center}
\textsf{\vfill}
\par\end{center}

\textsf{Diplomant: Bc. Martin Knapovský}

\textsf{Vedoucí diplomové práce: prof. Ing. Václav Řepa, CSc.}

\textsf{Oponent diplomové práce: Ing. Tomáš Bruckner, Ph.D. }
\textsf{\vfill}

\begin{center}
\textsf{\textbf{\small{}Školní rok 2016/2017}}
\par\end{center}{\small \par}

\newpage\thispagestyle{empty}

~

\vspace{\fill}

\textsf{\textbf{\Large{}Prohlášení}}\textsf{\vspace{12pt}}

\textsf{\hspace{30pt}}Prohlašuji, že jsem diplomovou práci zpracoval
samostatně a že jsem uvedl všechny použité prameny a literaturu, ze
kterých jsem čerpal.\textsf{ \\\\\\}

\textsf{\hspace{30pt}}V Praze, dne 6. prosince 2016\textsf{\hspace{\fill}}Martin Knapovský
\newpage\thispagestyle{empty}


\textsf{\textbf{\Large{}Poděkování}}\textsf{\vspace{12pt}}

\textsf{\hspace{30pt}}Rád bych tímto poděkoval prof. Ing.
Václavu Řepovi, CSc. za cenné rady při vedení mé práce. 

\newpage\thispagestyle{empty}

\noindent

\textsf{\textbf{\Large{}Abstrakt}}\vspace{12pt}
	
Cílem této diplomové práce je detailně popsat aktuální způsoby optimalizace webových stránek pro vyhledávače (SEO) a ukázat jejich aplikaci v širší marketingové strategii italské restaurace La Casa Degli Amici. Text předkládá důkazy o důležitosti SEO pro marketing a lze ho využít jako předlohu k optimalizaci vlastních webových stránek. Práce je rozdělena do dvou hlavních částí - části teoretické a praktické. Teoretická část vysvětluje koncepty použité při návrhu a implementaci v praktické části. V teoretické části se na vyhledávání díváme skrze 3 subjekty - vyhledávač, uživatele a SEO, které představuje tvorbu webových stránek a jejich optimalizaci. Je popsána historie vyhledávačů a způsob, kterým vyhledávače analyzují a ohodnocují webové stránky, dále se zabýváme vyhledávacími strategiemi uživatelů a způsobem, kterým čtou výsledky. V části o SEO jsou ukázány její výhody, přínosy, stručná historie, způsob tvorby SEO strategie a dále se text podrobně zabývá technikami a nástroji, které pro optimalizaci můžeme použít. V praktické části sestavujeme plán marketingové strategie pro italskou restauraci, pro kterou byl navržen a implementován vlastní web, na kterém byly použity vybrané techniky uvedené v teoretické části. Implementace marketingové strategie je vyhodnocena na datech získaných v období od spuštění webu (1. 8. 2016) do 31. 11. 2016. Majitel restaurace poskytl data o tržbách restaurace, díky kterým bylo možné také vyhodnotit celkový přínos implementované marketingové strategie. \\ 

\textbf{Klíčová slova:} SEO, Marketing, Vyhledávače, Google, Web, UX, PageRank, Responzivní design, Facebook, Wordpress\\

\textsf{\textbf{\Large{}Abstract}}\vspace{12pt}

The aim of this thesis is to describe in detail the current ways of optimizing websites for search engines (SEO) and to apply them in a broader marketing strategy of italian restaurant La Casa Degli Amici. The thesis presents evidence about the importance of SEO for marketing and can be used as a template for optimizing you own website. This thesis is divided into two main parts - theoretical and practical. The theoretical part explains the concepts used in the design and implementation of the practical part. The theoretical part considers 3 main subjects - search engine, users and SEO which represents website creation and optimization. This thesis describes the history of search engines and the way search engines analyze and evaluate a website. It also describes user search strategies and the way users read the search results. In the SEO section we show its advantages, brief history, SEO strategies and description of techniques and tools that can be used for SEO. In the practical part we put together a marketing strategy which consists of website creation, implementation of selected SEO techniques and social network marketing. Implementation of the marketing strategy is evaluated on data obtained during the period between 1. 8. 2016 and 31. 11. 2016. Owner of the restaurant has provided data on sales which were used for overall evaluation of implemented marketing strategy. \\

\textbf{Keywords:} SEO, Marketing, Search Engines, Google, Web, UX, PageRank, Responsive design, Facebook, Wordpress

\newpage\thispagestyle{empty}

\setcounter{secnumdepth}{2}
\setcounter{tocdepth}{2}
\tableofcontents{}

\listoffigures
\begingroup
\let\clearpage\relax
\listoftables
\endgroup
\begingroup
\let\clearpage\relax
\renewcommand{\lstlistlistingname}{Seznam zdrojových kódů}
\lstlistoflistings
\endgroup

\newpage\thispagestyle{empty}
